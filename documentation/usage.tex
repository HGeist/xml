\section{Benutzung}
Der POInter wird folgendermaßen benutzt:
Mit dem Öffnen der Url (http://cochu.dyndns.org/xml/) gelangt man zur Startseite des POInters.
Diese besteht aus einem Suchfeld und einem Button zum Starten der Suchanfrage.
Zusätzlich ist es möglich vor dem Abschicken der Suchanfrage mehrere Checkboxes auszuwählen,
um sich in der nähe des Tracks befindende POIs in der jeweiligen Kategorie anzuzeigen.
Generell unterscheidet die Suche bei dem Suchwort nicht in der Klein- und Großschreibung!
Nach dem Abschicken der Suchanfrage, sucht der Server das Resultat zusammen und zeigt dem Nutzer alle gefundenen
Tracks tabellarisiert an.
In dieser Tabelle ist der Track Titel und das Erstellungsdatum des Tracks vorhanden.
Klickt man nun auf einen Track, öffnet sich eine Karte mit der ausgewählten Strecke eingezeichnet.
Mit dem Klick auf den Startpunkt, erhält man sämtliche Informationen zum Track, außerdem werden alle
gefundenen POIs der gewünschten Kategorie, die zum Track zugeordnet sind, auf der Karte angezeigt.
