\documentclass[10pt,a4paper]{scrartcl}
\usepackage[ngerman]{babel}
\usepackage[utf8]{inputenc}
\usepackage[T1]{fontenc}
\usepackage{fancyhdr}
\usepackage{hyperref}
\usepackage{listings}
\usepackage[a4paper,inner=3.5cm,outer=3cm,top=4cm,bottom=3cm,pdftex]{geometry}

\setkomafont{sectioning}{\normalfont\bfseries}

\pagestyle{fancy}
\fancyhf{}
\renewcommand{\headrulewidth}{0.4pt}

\title{XML-Projekt SoSe 2012}
\subtitle{GPSies.org, DBPedia.org, Twitter.com, XML/XSD/XSLT\vspace*{.7cm}}
\author{Gruppe 11\\\hfill\\Eike Cochu, Samer El-Safadi, Hannes Geist,\\Cenk Gündogan, Michael Pluhatsch}
\date{\today}
\fancyfoot[C]{\thepage}
\fancyhead[L]{XML SoSe2012}
\fancyhead[R]{Cochu, El-Safadim, Geist, Gündogan, Pluhatsch, \today}

\setlength\parskip{\medskipamount}
\setlength\parindent{0pt}

\RequirePackage[usenames,dvipsnames]{color}
\RequirePackage{listings}
\lstset{
  language=Java,
  basicstyle=\ttfamily\small,
  numbers=left,
  numberstyle=\tiny,
  numbersep=5pt,
  tabsize=2,
  showstringspaces=false,
  extendedchars=true,
  breaklines=true,
  showtabs=false,
  showspaces=false,
  keywordstyle=\color{orange},
  commentstyle=\color{ForestGreen},
  stringstyle=\color{blue},
  escapechar=@,
  morekeywords={bool,iffalse}
}

\begin{document}
\maketitle
\thispagestyle{empty}
\vspace*{2cm}
\tableofcontents
\newpage
\renewcommand{\baselinestretch}{1.5}
\selectfont

\section{Aufgabenstellung und Organisation}
Gruppenleiter: Hannes Geist

\subsection{Aufgabenstellung}
Durch den XML-Endpoints von gpsies.com 110.000 Wanderstrecken laden, per XSLT-Schema in ein eigens entwickeltes XSD-Schema transformieren und in einer geeigneten XML-Datenbank speichern. Zu diesen Strecken soll per SPARQL der Endpoint von dbpedia.org abgefragt und alle auf (oder an) der Strecke liegenden Sehenswürdigkeiten abgefragt werden können. Ein HTML-Formular (oder ähnliches) entwickeln, mit dem die XML-Datenbank abgefragt werden kann. Das Ergebnis der Abfrage soll wiederum per XSLT in HTML transformiert und zusammen mit den vorhandenen Sehenswürdigkeiten und den zu diesen Sehenswürdigkeiten vorhandenen Tweets von Twitter.com angezeigt werden.

\subsection{Organisatorisches}
Zur Gruppenkoordinierung hat sich unsere Gruppe jede Woche Sonntags beim Gruppenleiter Hannes Geist getroffen und das weitere Vorgehen besprochen und gemeinsam an den einzelnen Softwarekomponenten gearbeitet. Diese Dokumentation ist gemeinsam entstanden, wobei jeder seinen Aufgabenteil beschrieben und etwaige Probleme, die im Verlauf der Bearbeitung auftraten, aufgezählt und ausgeführt hat.

Die endgültigen Ergebnisse sind momentan noch unter \href{http://cochu.dyndns.org/xml} erreichbar.

\subsection{Aufgabenverteilung und -bewältigung}
\begin{itemize}
\item Eike Cochu: Erstellung des XSD-Schemas, Bereitstellung eines Servers, Dokumentation
\item Samer El-Safadi: SPARQL-Abfrage der Sehenswürdigkeiten von dbpedia.org
\item Hannes Geist: XSLT-Transformierungen und Füllen der XML-Datenbank
\item Cenk Gündogan: Abfrage der Strecken von gpsies.org 
\item Michael Pluhatsch: HTML-Formular mit PHP, Abfrage der XML-Datenbank
\end{itemize}

% ------------------------------------------------------------------------------
% Teil mit den individuellen Aufgabenbeschreibungen, jeder hat seine eigene 
% .tex-Datei! Text dazu bitte nicht hier reinschreiben
% Inhalt eures Textes: Aufgaben, die ihr erledigt habt, wie und mit was ihr sie
% erledigt habt, was ihr für Probleme hattet usw.
\subsubsection{Eike Cochu}
Aufgabenbereich: Erstellung eines XSD-Schemas für die XML-Datenbank, Bereitstellung des Datenbank- und Applikationsservers sowie Installation der Datenbank, Einfügen der Inhalte in die Datenbank und Installation der Webseite, kleine kosmetische Änderungen an der Webseite und Formatierung, Erstellen der Dokumentation.

\paragraph{XSD-Schema}
Nachdem wir uns mit einem kurzen Testlauf einige Anschauungsdaten von gpsies.org beschafft hatten, konnten wir basierend darauf grob die Anforderungen an das XSD-Schema entwerfen. Die Idee des Schemas sollte es sein, mehr mit Attributen zu arbeiten und so die Anzahl an Elementen zu verringern, damit die Speichergröße der resultierenden Dateien minimal gehalten wird. Dazu haben wir uns entschlossen, einige weniger wichtige Datenteile der crawl-Daten garnicht erst mit in die Datenbank zu speichern, im Schema waren diese dann auch nicht vorgesehen. Die meisten Elemente sind mittels minOccurs=0 ebenfalls optional gehalten, Attribute sind per default optional.

Zur Veranschauung ein kleiner Ausschnitt aus dem Schema:

\begin{lstlisting}[language=XML]
<!-- Eine Adresse. Optional: alles -->
<xsd:complexType name="address">
  <xsd:attribute name="street" type="xsd:string"/>
  <xsd:attribute name="streetnumber" type="xsd:string"/>
  <xsd:attribute name="zipcode" type="xsd:string"/>
  <xsd:attribute name="city" type="xsd:string"/>
  <xsd:attribute name="country" type="xsd:string"/>
</xsd:complexType>
\end{lstlisting}

Hier ist zu sehen, dass in unserem Schema hauptsächlich Attribute verwendet werden, um die Anzahl an Elementen zu verringern. Auch sind Attribute standardmäßig optional, was bei Elementen erst hätte eingerichtet werden müssen.

Beim Schema sind wir auf keine Probleme gestoßen. Das XSD erlaubt es auf einfache Art und Weise ein sehr dynamisches Schema zu erstellen, dass wir im Prozess der Entwicklung aufgrund von neuen Erkenntnissen immer wieder verädert haben, bis es allen Anforderungen entsprach. Die Validierung des Schemas wurde mit einem gewöhnlichen XSD-Schemavalidierer durchgeführt, der unser Schema mit dem eigentlichen XSD-Schema validieren konnte.

\paragraph{Datenbank}
Wir haben uns für BaseX als Datenbank entschieden, da dieses auf Java basiert und somit systemunabhängig installierbar und auch leicht zu handhaben ist. Die Datenbank wurde auf einem dedizierten Ubuntu 12.04 Server installiert, als Webserver wurde Apache + PHP 5 gewählt.

Probleme mit der Datenbank: Da das entgültige Datenbankfile eine Größe von ~1.2 GB hatte und der Heapspeicher der JVM default sehr klein ist, trat beim Erstellen der Datenbank immer ein Out Of Memory Fehler auf, den ich erst mit dem manuellen hochsetzen des Heapspeichers beheben konnte ({\tt java -cp basex.jar org.basex.BaseXServer -Xmx1G}). Bei genauerer Untersuchung habe ich herausgefunden, dass dieser Speicherfehler speziell durch die Volltextindizierung ausgelöst wird, die auch gezieht ausgeschaltet werden kann. 

\subsubsection{Samer El-Safadi}
Bei der Aufgabe die XML-Datenbank um relevante "Points of interest"-Daten
zu erweitern, sind wir uns sehr schnell einig gewesen, uns auf Deutschland
zu konzentrieren und sämtliche andere Länder außer Acht zu lassen.\\

Um an jene Punkte zu kommen, mussten wir uns erst einmal über die
Abfragesprache SPARQL informieren. Dies haben wir größtenteils über das
Buch "Learning SPARQL" von Bob DuCharme getan, in dem sowohl allgemeine
Informationen über die Materie (sprich das "Semantic Web", RDF, Linked
Data etc.), als auch spezielle Situationen dargestellt sind, die über
SPAQL-Abfragen anschaulich bearbeitet werden. Praktischerweise beziehen
sich einige der Beispiele direkt auf DBpedia, wodurch wir das Geschriebene
auch gleich testen konnten.\\

Nachdem wir damit nun eigene Abfragekonstrukte erstellen konnten, die wir
über den Online-Zugriff ausprobiert haben, ging es nun darum den
SPARQL-Endpoint nun auch über Java ansprechen zu können. Schnell fiel die
Wahl auf das Jena Framework von Apache, welches uns einen sehr
komfortablen Weg ermöglichte über ARQ auf den entsprechenden Endpunkt
zuzugreifen: Wir konnten also durch die QueryFactory entsprechende
Anfragen erstellen und diese über QueryExecutionFactory.sparqlService an
einen definierten Service (in diesem Fall DBpedia) senden.\\

Das Programm machte uns durch Warnungen darauf aufmerksam, dass kein
Logger eingerichtet worden ist. Wir behoben dies gleich und integrierten
log4j (ebenfalls Apache), um die Ergebnisse der Anfrage in ein Logfile zu
schreiben, statt es nur (testweise) in die Standardausgabe zu packen.
Als dies erledigt war, konnten wir über die Ergebnisse (ResultSet)
iterieren (.hasNext() bzw. .next()) und diese dann über den Logger direkt
in eine Datei schreiben.

Da das Ergebnis allerdings ein XML-Dokument sein sollte, um es über XSLT
entsprechend einfach umzuwandeln, mussten wir erstmal eine Art Parser
schreiben, der über die Iteration der Ergebnisse ein wohl geformtes
XML-Dokument erstellt.

Dies funktionierte zwar, stellte sich allerdings als sehr aufwendig
heraus, so dass wir uns nach einer neuen Lösung umgesehen haben: Jenas
ResultSetFormatter.outputAsXML war die perfekte Lösung unserer Probleme.\\

Wir mussten entsprechend umdisponieren, haben den log4j-Logger wieder
entfernt und entsprechend mit dem Fileoutputstream gearbeitet, den wir
komfortabel über den ResultSetFormatter von Jena dazu bringen konnten uns
die Ergebnisse in eine sehr übersichtliche XML-Datei zu parsen.\\

Nun stand das Programm und es ging darum unsere Abfragekonstrukte durch
sinnvolle Abfragen zu ersetzen. Dafür mussten wir viel Zeit investieren,
um uns erst einmal einen Überblick zu verschaffen, wie die Datensätze in
der DBpedia angeordnet waren. Wir merkten schnell, dass die englische
DBpedia weitaus weniger relevante Daten über Deutschland enthielt, als die
deutsche - die allerdings wiederum weitaus weniger GEO-Datensätze über die
entsprechenden Objekte enthielt (vergleiche http://de.dbpedia.org/page/Brandenburger\_Tor\\
mit http://dbpedia.org/page/Brandenburg\_Gate).\\

Außerdem mussten wir feststellen, dass viele Objekte oftmals falsch bzw.
scheinbar keiner Regel folgend in entsprechende Typen eingeordnet wurden
und alles doch recht chaotisch war, sodass wir nicht einfach eine große
Anfrage starten konnten und die Ergebnisse anschließend anhand ihrer Typen
zuordnen konnten. Wir verbrachten viel Zeit damit die yago-Sprachklasse
bzw. die Kategorien von DBpedia zu studieren, um entsprechend gezielte
Anfragen zu stellen. Mindestvoraussetzung eines Objektes war für uns die
Bezeichnung, Geo-Daten in Form von long und lat sowie ein Link zum
entsprechenden Wikipedia-Artikel, damit der Nutzer sich weitere
Informationen einholen konnte.\\

Durch dieses Vorgehen waren wir in der Lage Anfragen zu entwickeln, die
relativ schnell abgearbeitet wurden und wir wussten genau, um welche
Untermenge es sich bei den Ergebnissen handelte, sodass wir dies
entsprechend in der Datenbank vermerken konnten. Dadurch konnten wir die
Website entsprechend so einrichten, dass der Nutzer es sich einzeln
aussuchen konnte, ob dieser Parks, Seen etc. angezeigt bekommen wollte
oder nicht.

\subsubsection{Hannes Geist}
\paragraph{Aufgabenbereich}
Erstellung des DB-XML-Dokuments aus den Daten des GPSies Crawlers per XSLT und Zusammenführung mit den Daten des POI-Crawlers per Java, Nutzung von SAX und StAX.

\paragraph{Das XSLT-Dokument}
Die Ausgangsdaten bestehen aus dem Document-Element "'crawl"' und dem einzigen Kindelement "'query"', welches dann jeweils einen "'track"'-Knoten als Ergebnis der GPSsies-Anfragen enthält. Die Informationen in einem Trackknoten liegen in einer flachen Hierarchie vor, es existieren keine Attribute und Kindknoten sind maximal noch einmal verschachtelt um bestimmte mehrere Werte einer bestimmten Track-Eigenschaft aufzuzählen.
Das Tranformationsdokument besteht aus einer einfachen Template zur Erfassung jedes track-Knotens im GPSies-Crawl und einem komplexeren Template zur Transformation eines solchen. Dieses komplexere track-Knoten-Template fügt dem zu schreibenden track-Knoten im neuen Dokument je nach Vorhandensein des Knotens in einem track-Knoten des Ausgangsdokuments die Attribute trackName, author, createTimestamp, totalLength, totalAscend, totalDescend, altitudeMaxHeightM, altitudeMinHeightM, altitudeDifferenceM, totalDescendM und quality, sowie die Kindelemente fileId, kmlLink, trackProperty, description, trackAttributes, trackCharacters, trackRoadbeds, trackRoads, trackTypes und points hinzu. Dem points-Knoten werden point-Knoten als Kindelemente zugeordnet, die aus der Aufzählung der GPS-Positions- und Höhenkoordinaten extraiert werden und die Attribute lon für die geographische Länge, lat für die geographische Breite und ele für die Höhe über Normalnull erhält.

\paragraph{Der GPSies-POI-Parser}
Es handelt sich um ein Java-Programm, das zur Ausführung als Eingabedateien die in der main-Methode im Feld poiFiles festgelegten XML-Dokumente, sowie die aus der oben beschriebenen XSL Transformation resultierende XML-Datei benötigt. Die main-Methode befindet sich in der Klasse "'Parser"'.

\subparagraph{Programmablauf}
Zuerst werden die aufgeführten POI-Dokumente über die Methode parsePois der Klasse PoiParser in DOM-Objekte geparst, die einer Hashmap "'pois"' mit dem Dateinamen als Schlüssel und ArrayListen von Poi-Objekten als Werten besteht. Die Klasse Poi bildet einen POI mit den Eigenschaften Titel(title), den Wikipedia-Link(wikiLink) und die GPS-Koordinaten(lon, lat), ohne Höhenangabe, eines POIs ab. Diese Vorgehensweise ist nur für eine relativ kleine Anzahl von POIs bzw. POI-Dokumenten sinnvoll, wie im aktuellen Stand unserer Software.
In einem zweiten Schritt wird nun die Methode connectPois eines Objekts der Parser-Klasse aufgerufen, die einem SAX-Parser-Objekt eine Instanz der Klasse PoiHandler übergibt, in der die eigentliche Zuordnungslogik von POIs zur GPSies-Datenbankdatei implementiert ist. 
Der PoiHandler erweitert den DefaultHandler des SAX-Parsers so, dass ein XML-Dokument mit dem Namen "'gpsies_pois.xml"' erzeugt wird. Das Document-Element heißt "'tracks"' und enthält track-Elemente. Deren points-Kindelemente mit den enthaltenen point-Elementen werden ausgelesen und ein Rechteck aus dem am weitesten südlich/westlich und dem am weitesten nördlich/östlich gelegenen Punkt + 10% der errechneten Rechteckhöhe bzw. -breite errechnet, ausgehend vom, bezüglich Gesamtzahl der Punkte, in der Mitte der Trackpunktmenge liegenden Trackpunkt.

Folgende Anmerkungen zur eingereichten Version: Die aktuelle Softwareversion enthält noch einen Test bzw. einen Fehler in der Methode connectPois, bei dem unter dem Namen der zu erzeugenden Ausgabedatei ein XML-Writer- und ein Dateistrom erstellt werden und mit einem leeren Element beschrieben werden. Im PoiHandler werden XML-Writer und Dateistrom am Ende des Parsevorgangs nicht explizit geschlossen, was im aktuellen Fall keine Fehler erzeugt, jedoch bei einer Weiterentwicklung zu Fehlern führen könnte. Außerdem enthält die Klasse im abgegebenen Quellcode das öffnende tracks-Element nicht. Anscheinend wurde vergessen die Änderungen zur Erzeugung des finalen Datenbankdokuments abzuspeichern bzw. wurde der Quellcode danach versehentlich mit einer älteren Version überschrieben (Das erklärt einige Missverständnisse, ich kann mir selbst grad nicht erklären was schief gelaufen ist und wo mein Code hin ist).
\subsubsection{Cenk Gündogan}

\subsubsection{Michael Pluhatsch}

Die Weboberfläche, index.html, besteht aus einem Formular mit einem Textfeld, in das man ein Stichwort eingeben kann. Nach diesem wird später in den Titeln der Tracks in der Datenbank gesucht. Man kann per Checkboxes auswählen welche POIs man sehen möchte und auf den Suchschalter klicken.

Die search.php erhält von der index.html per POST den Suchbegriff und startet eine Session und instaziiert ein Objekt der Hilfklasse Session aus BaseXClient.php, die die Kommunikation mit dem BaseX Server schachtelt. Der Datenbankserver BaseX wird nach track-Elementen abgefragt, bei denen das Suchwort im Kindelement trackName vorkommt. Die Ergebnisse werden als HTML-Tabelle mit den Spalten Track Title und CreateDate zurückgegeben.

Der jeweilige Track Title in der HTML-Tabelle enthält einen Link auf die search.php selbst mit der trackId als URL-Parameter. Klickt der Nutzer nun auf einen der Track-Links wird die search.php erneut mit der trackId als GET-Parameter aufgerufen und in der Funktion showTrack dieser Track noch einmal komplett vom Datenbankserver abgefragt. Aus dem Ergebnis wird eine Seite mit eingebundenen Google Maps Javascripten erzeugt und neben den bereits vorhandenen Markern und der Verbindungslinie zwischen ihnen aus dem mit dem Track verbunden KML-file werden die POIs, in dieser Version, als gewöhnliche Marker hinzugefügt. Der Nutzer sieht als Ergebnis eine Google Map, die den gesamten Track anzeigt und POI-Marker, deren Name beim Hovern mit der Maus über diesen angezeigt wird.


% ------------------------------------------------------------------------------

\section{Installation und Systemvoraussetzungen}

Zur Installation wird ein Webserver + PHP und BaseX als XML-Datenbankserver benötigt. Die web-Dateien gehören in ein Unterverzeichnis in den Documentroot des Webservers. Die Datenbank wird mittels
\begin{verbatim}
$ java -cp basex.jar org.basex.BaseXServer -Xmx1G
$ java -cp basex.jar org.basex.BaseXClient
$ > create db db-crawl <input-file.xml>
\end{verbatim}

erstellt und der Datenbankserver gestartet. Erst dann kann das PHP-Script der Hauptseite auf die Datenbank zugreifen.

Als Systemvoraussetzungen gelten eine Mindestmenge von 2GB RAM für das Erstellen und effiziente Durchsuchen der Datenbank.

\end{document}
