\subsubsection{Samer El-Safadi}
Bei der Aufgabe die XML-Datenbank um relevante "Points of interest"-Daten
zu erweitern, sind wir uns sehr schnell einig gewesen, uns auf Deutschland
zu konzentrieren und sämtliche andere Länder außer Acht zu lassen.\\

Um an jene Punkte zu kommen, mussten wir uns erst einmal über die
Abfragesprache SPARQL informieren. Dies haben wir größtenteils über das
Buch "Learning SPARQL" von Bob DuCharme getan, in dem sowohl allgemeine
Informationen über die Materie (sprich das "Semantic Web", RDF, Linked
Data etc.), als auch spezielle Situationen dargestellt sind, die über
SPAQL-Abfragen anschaulich bearbeitet werden. Praktischerweise beziehen
sich einige der Beispiele direkt auf DBpedia, wodurch wir das Geschriebene
auch gleich testen konnten.\\

Nachdem wir damit nun eigene Abfragekonstrukte erstellen konnten, die wir
über den Online-Zugriff ausprobiert haben, ging es nun darum den
SPARQL-Endpoint nun auch über Java ansprechen zu können. Schnell fiel die
Wahl auf das Jena Framework von Apache, welches uns einen sehr
komfortablen Weg ermöglichte über ARQ auf den entsprechenden Endpunkt
zuzugreifen: Wir konnten also durch die QueryFactory entsprechende
Anfragen erstellen und diese über QueryExecutionFactory.sparqlService an
einen definierten Service (in diesem Fall DBpedia) senden.\\

Das Programm machte uns durch Warnungen darauf aufmerksam, dass kein
Logger eingerichtet worden ist. Wir behoben dies gleich und integrierten
log4j (ebenfalls Apache), um die Ergebnisse der Anfrage in ein Logfile zu
schreiben, statt es nur (testweise) in die Standardausgabe zu packen.
Als dies erledigt war, konnten wir über die Ergebnisse (ResultSet)
iterieren (.hasNext() bzw. .next()) und diese dann über den Logger direkt
in eine Datei schreiben.

Da das Ergebnis allerdings ein XML-Dokument sein sollte, um es über XSLT
entsprechend einfach umzuwandeln, mussten wir erstmal eine Art Parser
schreiben, der über die Iteration der Ergebnisse ein wohl geformtes
XML-Dokument erstellt.

Dies funktionierte zwar, stellte sich allerdings als sehr aufwendig
heraus, so dass wir uns nach einer neuen Lösung umgesehen haben: Jenas
ResultSetFormatter.outputAsXML war die perfekte Lösung unserer Probleme.\\

Wir mussten entsprechend umdisponieren, haben den log4j-Logger wieder
entfernt und entsprechend mit dem Fileoutputstream gearbeitet, den wir
komfortabel über den ResultSetFormatter von Jena dazu bringen konnten uns
die Ergebnisse in eine sehr übersichtliche XML-Datei zu parsen.\\

Nun stand das Programm und es ging darum unsere Abfragekonstrukte durch
sinnvolle Abfragen zu ersetzen. Dafür mussten wir viel Zeit investieren,
um uns erst einmal einen Überblick zu verschaffen, wie die Datensätze in
der DBpedia angeordnet waren. Wir merkten schnell, dass die englische
DBpedia weitaus weniger relevante Daten über Deutschland enthielt, als die
deutsche - die allerdings wiederum weitaus weniger GEO-Datensätze über die
entsprechenden Objekte enthielt (vergleiche http://de.dbpedia.org/page/Brandenburger\_Tor\\
mit http://dbpedia.org/page/Brandenburg\_Gate).\\

Außerdem mussten wir feststellen, dass viele Objekte oftmals falsch bzw.
scheinbar keiner Regel folgend in entsprechende Typen eingeordnet wurden
und alles doch recht chaotisch war, sodass wir nicht einfach eine große
Anfrage starten konnten und die Ergebnisse anschließend anhand ihrer Typen
zuordnen konnten. Wir verbrachten viel Zeit damit die yago-Sprachklasse
bzw. die Kategorien von DBpedia zu studieren, um entsprechend gezielte
Anfragen zu stellen. Mindestvoraussetzung eines Objektes war für uns die
Bezeichnung, Geo-Daten in Form von long und lat sowie ein Link zum
entsprechenden Wikipedia-Artikel, damit der Nutzer sich weitere
Informationen einholen konnte.\\

Durch dieses Vorgehen waren wir in der Lage Anfragen zu entwickeln, die
relativ schnell abgearbeitet wurden und wir wussten genau, um welche
Untermenge es sich bei den Ergebnissen handelte, sodass wir dies
entsprechend in der Datenbank vermerken konnten. Dadurch konnten wir die
Website entsprechend so einrichten, dass der Nutzer es sich einzeln
aussuchen konnte, ob dieser Parks, Seen etc. angezeigt bekommen wollte
oder nicht.
