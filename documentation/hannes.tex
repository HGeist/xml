\subsubsection{Hannes Geist}
\paragraph{Aufgabenbereich}
Erstellung des DB-XML-Dokuments aus den Daten des GPSies Crawlers per XSLT und Zusammenführung mit den Daten des POI-Crawlers per Java, Nutzung von SAX und StAX.

\paragraph{Das XSLT-Dokument}
Die Ausgangsdaten bestehen aus dem Document-Element "'crawl"' und dem einzigen Kindelement "'query"', welches dann jeweils einen "'track"'-Knoten als Ergebnis der GPSsies-Anfragen enthält. Die Informationen in einem Trackknoten liegen in einer flachen Hierarchie vor, es existieren keine Attribute und Kindknoten sind maximal noch einmal verschachtelt um bestimmte mehrere Werte einer bestimmten Track-Eigenschaft aufzuzählen.
Das Tranformationsdokument besteht aus einer einfachen Template zur Erfassung jedes track-Knotens im GPSies-Crawl und einem komplexeren Template zur Transformation eines solchen. Dieses komplexere track-Knoten-Template fügt dem zu schreibenden track-Knoten im neuen Dokument je nach Vorhandensein des Knotens in einem track-Knoten des Ausgangsdokuments die Attribute trackName, author, createTimestamp, totalLength, totalAscend, totalDescend, altitudeMaxHeightM, altitudeMinHeightM, altitudeDifferenceM, totalDescendM und quality, sowie die Kindelemente fileId, kmlLink, trackProperty, description, trackAttributes, trackCharacters, trackRoadbeds, trackRoads, trackTypes und points hinzu. Dem points-Knoten werden point-Knoten als Kindelemente zugeordnet, die aus der Aufzählung der GPS-Positions- und Höhenkoordinaten extraiert werden und die Attribute lon für die geographische Länge, lat für die geographische Breite und ele für die Höhe über Normalnull erhält.

\paragraph{Der GPSies-POI-Parser}
Es handelt sich um ein Java-Programm, das zur Ausführung als Eingabedateien die in der main-Methode im Feld poiFiles festgelegten XML-Dokumente, sowie die aus der oben beschriebenen XSL Transformation resultierende XML-Datei benötigt. Die main-Methode befindet sich in der Klasse "'Parser"'.

\subparagraph{Programmablauf}
Zuerst werden die aufgeführten POI-Dokumente über die Methode parsePois der Klasse PoiParser in DOM-Objekte geparst, die einer Hashmap "'pois"' mit dem Dateinamen als Schlüssel und ArrayListen von Poi-Objekten als Werten besteht. Die Klasse Poi bildet einen POI mit den Eigenschaften Titel(title), den Wikipedia-Link(wikiLink) und die GPS-Koordinaten(lon, lat), ohne Höhenangabe, eines POIs ab. Diese Vorgehensweise ist nur für eine relativ kleine Anzahl von POIs bzw. POI-Dokumenten sinnvoll, wie im aktuellen Stand unserer Software.
In einem zweiten Schritt wird nun die Methode connectPois eines Objekts der Parser-Klasse aufgerufen, die einem SAX-Parser-Objekt eine Instanz der Klasse PoiHandler übergibt, in der die eigentliche Zuordnungslogik von POIs zur GPSies-Datenbankdatei implementiert ist. 
Der PoiHandler erweitert den DefaultHandler des SAX-Parsers so, dass ein XML-Dokument mit dem Namen "'gpsies_pois.xml"' erzeugt wird. Das Document-Element heißt "'tracks"' und wird wie alle anderen Elemente des Ausgangsdokuments in der Handler-Methode "'endElement"' automatisch in das neue Ausgabedokument geschrieben. Eine Ausnahme bilden die track-Elemente, die in der genannten Methode eine Sonderbahndlung erfahren. Deren points-Kindelemente mit den enthaltenen point-Elementen werden ausgelesen und ein Rechteck aus dem am weitesten südlich/westlich und dem am weitesten nördlich/östlich gelegenen Punkt + 10% der errechneten Rechteckhöhe bzw. -breite errechnet, ausgehend vom, bezüglich Gesamtzahl der Punkte, in der Mitte der Trackpunktmenge liegenden Trackpunkt.

Folgende Anmerkungen zur eingereichten Version: Die aktuelle Softwareversion enthält noch einen Test bzw. einen Fehler in der Methode connectPois, bei dem unter dem Namen der zu erzeugenden Ausgabedatei ein XML-Writer- und ein Dateistrom erstellt werden und mit einem leeren Element beschrieben werden. Im PoiHandler werden XML-Writer und Dateistrom am Ende des Parsevorgangs nicht explizit geschlossen, was im aktuellen Fall keine Fehler erzeugt, jedoch bei einer Weiterentwicklung zu Fehlern führen könnte.